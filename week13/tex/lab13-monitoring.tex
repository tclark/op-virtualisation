\documentclass{article}
\usepackage{graphicx}
\usepackage{wrapfig}
%\usepackage{inconsolata}
\usepackage{enumerate}
\usepackage{hyperref}
\usepackage{verbatim}
\usepackage[parfill]{parskip}
\usepackage[margin = 2.5cm]{geometry}

\usepackage[T1]{fontenc}


\begin{document}

\title{Lab 13: Monitoring\\ID720 Virtualisation}
\date{}
\maketitle

\section*{Introduction}
As we move our applications into containers running in an environment like Kubernetes, we need some extra machinery to give us some visibility into our systems. \emph{Prometheus} is a typical tool for this.

Note that this lab is an updated and simplified version of the tutorial found at \url{https://cloud.google.com/architecture/monitoring-apps-running-on-multiple-gke-clusters-using-prometheus-and-stackdriver} 

\section{Setup}

To begin, create a project called ``lab13'' in GCP, and be sure that GKE and monitoring are enabled for this project.

Clone the files needed for this lab from \url{https://github.com/tclark/lab13-files} into your cloud shell.

In your cloud shell, you will need to install \emph{Helm}, a package manager for Kubernetes app. Do this by checking \url{https://helm.sh/docs/intro/install/} and following the directions for installing with Apt.

Launch a cluster:

\begin{verbatim}
  gcloud container clusters create lab13 \
  --zone us-west2-a \
  --num-nodes 3 \
  --machine-type n1-standard-2
\end{verbatim}

\section{Create Service Account}
You will need a service account with permissions to write data to the GCP monitoring service. Create the account with 

\begin{verbatim}
 gcloud iam service-accounts create prometheus --display-name prometheus-service-account
\end{verbatim}

and then save some values in environment variables

\begin{verbatim}
 export PROJECT_ID=$(gcloud info --format='value(config.project)')
 
 export PROMETHEUS_SA_EMAIL=$(gcloud iam service-accounts list \
    --filter="displayName:prometheus-service-account" \
    --format='value(email)')
\end{verbatim}

and finally give the service account a role to write metrics data

\begin{verbatim}
  gcloud projects add-iam-policy-binding ${PROJECT_ID} \
  --role roles/monitoring.metricWriter \
  --member serviceAccount:${PROMETHEUS_SA_EMAIL}
\end{verbatim}

\section{Install Prometheus}
Be sure to \texttt{cd} into the \texttt{lab13-files} directory.

Create a namespace for the prometheus resources

\begin{verbatim}
  kubectl create namespace prometheus
\end{verbatim}  

Create a Kubernetes service account associated with the service account you created above.

\begin{verbatim}
  kubectl apply -f prometheus-service-account.yaml
\end{verbatim}

  
Load the configMap that will supply configuration to the Prometheus server.

\begin{verbatim}
  kubectl apply -f prometheus-service-account.yaml
\end{verbatim}

Set up some environment variables that we will apply to the Prometheus deployment.

\begin{verbatim}
 export KUBE_NAMESPACE=prometheus
 export KUBE_CLUSTER=lab13
 export GCP_LOCATION=us-west2-a
 export GCP_PROJECT=$(gcloud info --format='value(config.project)')
 export DATA_DIR=/prometheus
 export DATA_VOLUME=prometheus-storage-volume
 export SIDECAR_IMAGE_TAG=0.8.2
 export PROMETHEUS_VER_TAG=v2.19.3
\end{verbatim} 

Now we combine these values into our deployment config and apply it.

\begin{verbatim}
  envsubst < gke-prometheus-deployment.yaml | kubectl apply -f -
\end{verbatim}

After a few minutes, verify that the Prometheus deployment is ready.

\begin{verbatim}
  kubectl get pods -n prometheus
\end{verbatim}

\section{Check the Prometheus UI}
Prometheus' role is to collect data from targets in our cluster and store the data in a time series database. We'll use that data to populate our dashboard later. First let's inspect Prometheus' web UI.

Set up port forwarding.
\begin{verbatim}
  export PROMETHEUS_POD_GKE=$(kubectl get pods \
  --namespace prometheus -l "app=prometheus-server" \
    -o jsonpath="{.items[0].metadata.name}")
    
  kubectl --context gke port-forward --namespace prometheus\
  $PROMETHEUS_POD_GKE 9090:9090 >> /dev/null &
\end{verbatim}

In cloud shell, click web preview > change port and enter 9090 for the new port number. Click ``Change and preview''. You should see the Prometheus web UI. In it, click Status > Service Discovery to see the services that Prometheus has found.

\section{Install Postgres}
We will use Postgresql to give us something interesting to monitor. We will install it with Helm.

First, add a repository and update Helm.
\begin{verbatim}
  helm repo add bitnami https://charts.bitnami.com/bitnami
  helm repo update
\end{verbatim}

Next, install Postgres.
\begin{verbatim}
  helm install pg bitnami/postgresql \
  --set metrics.enabled=true \
  --set postgresqlDatabase=prometheusdb \
  --set auth.database=labdb 
\end{verbatim}

Finally, prepare the database for later testing.

\begin{verbatim}
# get the pg password from a k8s secret
export POSTGRES_PASSWORD=$(kubectl get secret pg-postgresql -o 
jsonpath="{.data.postgres-password}" | base64 --decode)

# port-forward the database
kubectl --context gke port-forward \
svc/gke-postgresql 5432:5432 >> /dev/null &

# log into the database
 PGPASSWORD="$POSTGRES_PASSWORD" psql --host 127.0.0.1 -p 5432 -U postgres 

# in psql, create a database
CREATE DATABASE promtest
\end{verbatim}

\section{Create a Monitoring Dashboard}
In the cloud console, select Monitoring from the main menu (you'll have to scroll down a bit). Once it's loaded, choose the Metrics explorer.

\begin{enumerate}
  \item In the Find resource type and metric field, enter
  \texttt{pg\_stat\_database\_blks\_read}.
  \item Select the metric that starts with 
  \texttt{external.googleapis.com/prometheus/}
  \item For the resource, select Kubernetes Container
  \item In the Group By drop-down list, select \texttt{cluster\_name}.
  \item For Aggregator, select \texttt{sum}.
  \item Click Save Chart. Name the chart.
  \item In the Dashboard drop-down list, select New Dashboard.
  Enter the dashboard name Postgres and click Save.
  \item Click the Dashboards link, and then click the PostgreSQL dashboard.
\end{enumerate}

You should see a dashboard that plots a graph with our Postgres metric.

\section{Generate Postgres Traffic}
Use the tool \texttt{pgbench} to generate traffic on the database. Install it, if necessary, by installing the \texttt{postgresql-contrib} package.

In your cloud shell initialise the database with 

\begin{verbatim}
  PGPASSWORD="$POSTGRES_PASSWORD" pgbench -i \
  -h localhost -p 5431 -U postgres -d promtest
\end{verbatim}

And then run the following test, which will run for ten minutes.
\begin{verbatim}
  PGPASSWORD="$POSTGRES_PASSWORD" pgbench -c 10 -T 600 \
  -h localhost -p 5431 -U postgres -d promtest
\end{verbatim}

While this is running, check the monitoring dashboard. You should see the load from the test on the graph. Toggle auto-refresh on so that the graph will refresh.

\section{Clean up}
This lab uses a number of billable resources, so be sure to remove everything after you finish. The easiest way to do this is to delete the entire lab13 project.
 
\end{document}