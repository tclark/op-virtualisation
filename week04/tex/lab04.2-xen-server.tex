\documentclass{article}
\usepackage{graphicx}
\usepackage{wrapfig}
%\usepackage{inconsolata}
\usepackage{enumerate}
\usepackage{hyperref}
\usepackage{verbatim}
\usepackage[parfill]{parskip}
\usepackage[margin = 2.5cm]{geometry}

\usepackage[T1]{fontenc}


\begin{document}

\title{Lab 04.2: Xen Server\\ ID720 Virtualisation}
\date{}
\maketitle

\section*{Introduction}
In this lab we will get some experience using Xen with additional management tools on top of it. One thing to keep in mind is that behind the management interface we still have a Xen server. Everything we have done up to this point still works.

Xen Server is a special distribution of Xen with some extra tooling added, including XAPI. It can be used with a Windows 
based management tool. 

You need to be on the internal Polytech network to do this lab, and since you have to install a piece of software it is recommended that you use a lab machine.
 
 \section{Install Citrix Xen Center}
 Xen Center is the management tool we will use for this lab. An installer is at \\
 \url{https://www.citrix.com/downloads/citrix-hypervisor/product-software/xenserver-70-standard-edition.html}. Use it to install
 Xen Center and start it. It will appear in your start menu under ``Citrix''. When it starts, select ``Add a Server'' and connect to our lab server with the parameters that you'll find on the class Teams channel  
  
  
  \section{Examine the server console}
 
 Once you connect, select ``Objects'' from the lower left menu and find our server, \texttt{xenserver}, in the tree 
 view. When you select it, examine the various tabs in the main part of the window to see what options are available. 
 Select the Console tab and log into the management console with our usual password. Use \texttt{xl list} to see the guest on the host.
 
 
\section{Create a VM Guest} 
 To create a new VM, we need a \emph{template} and some installation media. Expand the Templates section of the 
 tree menu and  and select the Ubuntu 14.04 64 bit template. Why this one? Because it's the only one for which we have installation media. Right-click the template and run the New VM Wizard. When you get to the Installation Media selection, You should be able to select an Ubuntu 14.04 image from the Xenserver's local repository, and then select the install media from the server DVD drive.
 
 Once you create the VM, find it in the VM section of the tree menu, select your VM and go to the console tab. You'll be presented with the standard Ubuntu installation wizard. Generally just take the default options in the wizard. When you get to the Network section, choose Network 0.
 Our new VM will appear in the VMs section of the tree menu, Choose yours and pick the console tab. In it, work through 
 the Ubuntu installer. You may run into problems with DHCP on your guest, if so don't worry about it; it's not necessary
 for this lab.

\section{Install XenServer Tools}
XenServerTools is a software package we install on the guest to support extra control and monitoring functions. Once 
our server is installed and running, click the XenServer tools link on the General tab and work though the installation 
process. The documentation to do this is linked in the first dialogue.

After installing the package and rebooting the guest, you should see some additional information, particularly on the General tab for your guest.

\section{Attach a disk}
From the storage tab, click the ``Add'' button and work through the Add Virtual Disk wizard. Once this has been done,
note that our VM has a new ``disk'' attached at \texttt{/dev/xvdb}. To use this disk we would need to format it with a file 
system and mount it on the VM.

Just to see the difference, delete the new disk on the Storage tab and verify that \texttt{/dev/xvdb} is removed from the 
VM.


\section{Take a Snapshot}
One feature of VMs is that we can take \emph{snapshots} that let us easily back up our VM's. Make a snapshot of our VM from the Snapshots tab. Once this is done do something dumb like removing the user's home directory on the VM, then use the ``Revert To'' option to roll back to the snapshot version and check that the directory is restored.

\section{Shut Down and Remove VM}
Once you finish the lab, please shut down and delete your VM to conserve system resources. We are doing this on old
hand-me-down hardware.




\end{document}