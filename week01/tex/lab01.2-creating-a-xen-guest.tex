\documentclass{article}
\usepackage{graphicx}
\usepackage{wrapfig}
%\usepackage{inconsolata}
\usepackage{enumerate}
\usepackage{hyperref}
\usepackage{verbatim}
\usepackage[parfill]{parskip}
\usepackage[margin = 2.5cm]{geometry}

\usepackage[T1]{fontenc}


\begin{document}

\title{Lab 01.2: Creating a Xen Guest\\ ID720 Virtualisation}
\date{}
\maketitle

\section*{Introduction}
In this lab we will create a Xen guest domain and run a virtual machine.  A Xen guest has three components:

\begin{enumerate}
  \item A ``disk'' for the guest system;
  \item An operating system kernel for the hypervisor to run;
  \item A configuration file.
\end{enumerate}

We put the word ``disk'' in quotes above because while it can be a physical disk, it probably isn't. It may be a disk partition or it may be a file that acts as a sort of virtual disk. In today's lab we will use a partition. The operating system kernel may be placed on the disk, but it does not have to be.

Our process for creating a guest will be the following:

\begin{enumerate}
  \item Create a disk partition for the guest;
  \item Download installation media (i.e., our first kernel);
  \item Write a configuration file;
  \item Run the VM and install the operating system on the guest's disk;
  \item Modify the configuration file (because our kernel has changed);
  \item Run the final version of the VM.
 \end{enumerate}
 
 \textbf{N. B.:} The commands shown in this lab need root privileges, so either preface them with \texttt{sudo} or start a privileged session with the command 
 \texttt{sudo -i}.
 
 \section{Create a disk partition}
 We set us our Xen servers with the Logical Volume Manager (LVM) which lets us add, remove, and resize partitions easily. We will use that capability to create a partition to serve as the ``disk'' for our guest. LVM organises \emph{logical volumes} (partitions, basically) into \emph{volume groups}.
 
 Find the name of our volume group with the command
 
 \texttt{vgs}
 
 For example, if a xen host is named \texttt{triton}, then its volume group is \texttt{triton-vg}.
 Now create a new volume for our guest with the command 
 
 \texttt{lvcreate -L 10G -n lab2\_vol /dev/<volume group name>}
 
 This will give us a volume named \texttt{lab2\_vol}.

 \section{Download installation media}
 We will download the netboot installer for Ubuntu 14.04 from a nearby mirror with the commands
 
 \begin{verbatim}
 mkdir -p /var/lib/xen/images/ubuntu-netboot/trusty
 cd /var/lib/xen/images/ubuntu-netboot/trusty
 wget http://ucmirror.canterbury.ac.nz/ubuntu/dists/trusty/main/installer-amd64/
   current/images/netboot/xen/vmlinuz
 wget http://ucmirror.canterbury.ac.nz/ubuntu/dists/trusty/main/installer-amd64/
   current/images/netboot/xen/initrd.gz
 \end{verbatim}
 
 Note that I broke up those urls into two lines for readability, but you should enter them on one line each.
 
 \section{Write a configuration file}
 Xen needs a configuration file to inform it about how our VM should be set up. There are some example files in our installation
 and we can use one of those as a starting point.
 
 \begin{verbatim}
 cd /etc/xen
 cp xlexample.pvlinux lab1.2.cfg
 \end{verbatim}
 
 Now we just need to edit that file to add or modify some entries. In particular
 
 \begin{itemize}
  \item Set the \texttt{name} to \texttt{lab1.2vm}
  \item Set the \texttt{kernel} to \texttt{/var/lib/xen/images/ubuntu-netboot/trusty/vmlinuz}
  \item Set the \texttt{ramdisk} to \texttt{/var/lib/xen/images/ubuntu-netboot/trusty/initrd.gz}
  \item Set the \texttt{memory} to \texttt{1024}
   \item Set the \texttt{vcpus} to \texttt{1}
   \item Set the \texttt{disk} to \begin{verbatim}['/dev/<volume-group-name>/lab_1_2_vol,raw,xvda,rw']\end{verbatim}
 \end{itemize}
 
 \section{Install Ubuntu on your VM}
 Now we are ready to actually run the VM. Enter the command
 
 \texttt{xl create -c /etc/xen/lab1.2.cfg}
 
 to start the guest and also to attach to its console.
 
 Now we just have to run through the typical installation process. Set the hostname of your VM to \texttt{lab1.2}. When setting up the partions,
 choose use the entire disk (guided) without LVM. When the installation gets to the reboot phase you will be disconnected from the console and dropped back at your 
 Dom0 system's shell.
 
 \section{Modify the configuration}
 Before we can use the VM we need to change the configuration since its kernel parameter still directs it to use the installer's kernel. Comment out the \texttt{kernel} and \texttt{ramdisk} lines and add a line to use Xen's \texttt{pygrub} bootloader by adding the line
 
 \texttt{bootloader = "/usr/lib/xen-4.4/bin/pygrub"}
 
 Then restart our VM with the commands
 
 \begin{verbatim}
 sudo xl shutdown lab1.2vm
 sudo xl create -c /etc/xen/lab1.2.cfg
 \end{verbatim}
 
 You may need to hit the enter key a couple of times to get a login prompt at the console. From here we will want to explore a bit about how to use \texttt{xl} commands to manage guests. See \url{http://xenbits.xen.org/docs/4.4-testing/man/xl.1.html} for documentation. As a quick reference, you can disconnect from a guest console with \texttt{ctrl-]} and attach to one with the command \texttt{xl console <vm name>}.
 
 
 
\end{document}