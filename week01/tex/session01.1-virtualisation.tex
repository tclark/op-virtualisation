% Beamer slide template prepared by Tom Clark <tom.clark@op.ac.nz>
% Otago Polytechnic
% Dec 2012

\documentclass[10pt]{beamer}
\usetheme{Dunedin}
\usepackage{graphicx}
\usepackage{fancyvrb}
\usepackage{hyperref}

\newcommand\codeHighlight[1]{\textcolor[rgb]{1,0,0}{\textbf{#1}}}

\title{Actual Virtualisation, For a Start}

\author[ID720]{Virtualisation}
\institute[Otago Polytechnic]{
  Otago Polytechnic \\
  Dunedin, New Zealand \\
}
\date{}
\begin{document}



%----------- titlepage ----------------------------------------------%
\begin{frame}[plain]
  \titlepage
\end{frame}


\begin{frame}
  \frametitle{Weird idea} 
  We will start our exploration of virtualisation by talking about running ``proper'' virtual machines on top of some hardware.
  \end{frame}
  
\begin{frame}
  \frametitle{Basic idea}
  
  The main idea is that we run a piece of software, generally called a \emph{hypervisor} on top of our 
  hardware that provides an environment in which guest operating systems can run.
   
  However, there are several variations on this theme.
\end{frame}
  
\begin{frame}
  \frametitle{Emulation}
  
   With \emph{emulation} we create a virtual environment that models a hardware architecture that is 
   different from the underlying host. We do this to develop software for hardware that does not yet exist or
   to run old software build for an architecture that is no longer available.
\end{frame}

\begin{frame}
  \frametitle{Full virtualisation}
  
   \emph{Full virtualisation} presents guests with a virtual environment that matches the target architecture of the underlying hosts. It models the hardware layer completely so that the guest operating system doesn't require any modification to run in the virtual environment.
\end{frame}

\begin{frame}
  \frametitle{Paravirtualisation}
  
   \emph{Paravirtualisation} also presents guests with a virtual environment that matches the target architecture of the underlying hosts, but unlike full virtualisation,
   paravirtualisation requires that guest operating systems are modified to adapt to the virtual environment. This typically offers some performance advantages 
   over full virtualisation.
   
\end{frame}

\begin{frame}
  \frametitle{Hosted vs. bare metal}
  
  Another dimension to consider is whether the hypervisor is \emph{hosted} or \emph{bare metal}.
  
  \begin{itemize}
    \item A bare metal hypervisor runs directly on the host hardware, without an operating system.
    \item A hosted hypervisor runs on top of a full operating system.
    \end{itemize}
    
    These are also referred to as type 1 and type 2 virtualisation, respectively, which is stupid.
   
\end{frame}
   
\begin{frame}
  \frametitle{Xen}
   
   We are going to work with the Xen hypervisor over the next few weeks. Xen is an open source hypervisor that originated at the University of Cambridge in 2001. 
   It provides a paravirtualised (sometimes full) environment for x86 architectures.
     
\end{frame}

\begin{frame}
  \frametitle{Protection rings}
   
   The x86 architecture supports four \emph{protection rings}. Usually the operating system runs at ring 0 and user applications run at ring 3. 
   
    \vspace{5mm}
   
   With Xen, the hypervisor 
   runs at ring 0, guest operating systems at ring 1, and applications at ring 3. Since full access to the hardware is only available at ring 0, this lets the hypervisor 
   control resources and isolate guest domains from each other.
           
\end{frame}

\begin{frame}
  \frametitle{Domain 0}
   
  Xen guests run in \emph{domains}. In order to provide a management interface there is always one special domain, \emph{Domain0} or just 
  \emph{Dom0}. In Dom0 we (typically) run a Linux operating system that itself runs services that we use to work with the hypervisor and its guest domains. Dom0 is 
  started as soon as a Xen host is booted.
  
  \vspace{5mm}
  
  Regular guests run in unprivileged domains, usually called \emph{DomU}.
           
\end{frame}

\begin{frame}
  \frametitle{Device drivers}
  
  Dom0 has direct access to the physical hardware and has native drivers for it. It then presents simplified abstract drivers to DomU guests. Alternatively, Dom0 can delegate device access to special \emph{driver domains} to manage the actual hardware drivers. 
  
  \vspace{5mm}
             
\end{frame}

\begin{frame}
  \frametitle{XL}
  
  Dom0 runs a special tool stack \texttt{XL}.  \texttt{XL} works with the hypervisor to start, stop, and manage guest domains. 
  We typically interact with \texttt{XL} using \texttt{xl}, the Xen management interface.         
  
  \end{frame}


\begin{frame}
  \frametitle{The plan from here}
  
  Now you be assigned a physical machine on which you will install Xen. Over the next few weeks we will see how to configure Xen and use it to run virtual machines. 
  
\end{frame}
\end{document}
