\documentclass{article}
\usepackage{graphicx}
\usepackage{wrapfig}
%\usepackage{inconsolata}
\usepackage{enumerate}
\usepackage{hyperref}
\usepackage{verbatim}
\usepackage[parfill]{parskip}
\usepackage[margin = 2.5cm]{geometry}

\usepackage[T1]{fontenc}


\begin{document}

\title{Lab 7.1: Terraform \\ ID720 Virtualisation}
\date{}
\maketitle

\section*{Introduction}
We have been learning about the OpenStack API and accessing it with the 
Python SDK. One thing we keep in mind is that \emph{any} cloud service 
will have similar tools available. Given that, the next logical step is
to ask if we could have a general tool that abstracts out the differences
between these various platforms. Then we could write out one description of
some virtual machines we want to run and apply it to any cloud platform.

It turns out that we can have tools that are \textbf{close} to meeting that
goal, but they don't quite work on any platform. There are too many differences
between these platforms to allow us to write one script and run it anywhere, 
but we get pretty close. One such tool is called \emph{Terraform} 
(\url{https://www.terraform.io}). In this lab we will write a simple Terraform
configuration and use it to create and destroy a VM instance on OpenStack.

All of the steps below must be carried out on our \texttt{cloudapi} server.

\section{Setup}
When we run Terraform it will read all the files in the working directory
ending in \texttt{.tf}, so start by creating a subdirectory of your home 
directory called \texttt{terraform}. Place any files for this lab in that 
directory.

When using other OpenStack tools we loaded credentials from either an
RC file or a \texttt{clouds.yaml} file. For today's lab we will use
both. (This doesn't exactly make sense, but it does work.) Your OpenStack RC file 
may still be in your home directory, or you can download a new one. Copy the 
\texttt{clouds.yaml} file from your Python code directory into your new Terraform 
directory.

\section{Configuring our provider}
Create a new file in your Terraform directory named \texttt{lab7.tf}. This file 
will contain two elements, a \emph{provider} and a \emph{resource}.

A Terraform provider tells Terraform what sort of cloud service you are using. 
Our provider configuration is simple, and looks like this:

\begin{verbatim}
 terraform {
 required_version = ">= 0.14.0"
  required_providers {
    openstack = {
      source  = "terraform-provider-openstack/openstack"
      version = "~> 1.35.0"
      }
    }
  }

provider "openstack" {
  cloud = "catalystcloud"
}
\end{verbatim}
The first section tells terraform that we will use the OpenStack provider and provides details about the provider resources that will need to be downloaded from the provider repository.

The next section tells Terraform to use this provider, and 
tells it to look for the ``\texttt{catalystcloud}'' cloud in our \texttt{clouds.yaml} file.  Once you've added this element, 
save your file and return to your shell inside your Terraform directory.

\section{Initialising the project}
Now that we have some minimal configuration, we can initialise our project. First, don't forget to 
activate your RC file with the command

\begin{verbatim}
 source ~/tom-clark-openrc.sh
\end{verbatim}

to populate some needed environment variables. Now, run the command

\begin{verbatim}
 terraform init
\end{verbatim}

This will create a \texttt{.terraform} directory inside our project directory, and will also ensure that 
the OpenStack plugin is installed since we indicates that we are using an OpenStack provider.

\section{Configure a resource}
To actually manage OpenStack items with Terraform, we need to define \emph{resources}. For
this lab we will configure a resource defining a virtual machine instance. Add the following 
to your \texttt{.tf} file.

\begin{verbatim}
 resource "openstack_compute_instance_v2" "<your username>" {
  name = "<your username>"
  image_id = "2b3ca33a-7587-49e0-95b8-c22206f74481"
  flavor_id = "99fb31cc-fdad-4636-b12b-b1e23e84fb25"
  key_pair = "<your keypair name>"
  security_groups = ["default"]
  network {
    name = "private-net"
  }
}
\end{verbatim}

This configuration tells Terraform that we want a compute instance. The image id and flavor id are the GUIDs
of items defined in our OpenStack instance and were found be checking the web dashboard. You may 
should verify that those are correct values for the Ubuntu 10.04 minimal image and the \texttt{c1.c1r05}
flavor.

Once this resource is defined in your file, return to your shell and run the command

\begin{verbatim}
 terraform apply
\end{verbatim}

Terraform will display it's \emph{plan} for what changes will be made to our OpenStack project
to satisfy the configuration. Once you say ``yes'' to this proposed plan Terraform will carry out 
the planned steps. Once this is done, check the web dashboard to verify that your VM instance
was created.

Next re-run \texttt{terraform apply} and check the plan. Since the VM instance you configured is already
running, Terraform concludes that there is nothing to do.

\section{Clean up}
I keep reminding people to not leave unneeded items running in OpenStack. Terraform makes it easy in this case.
Simply run 

\begin{verbatim}
 terraform destroy
\end{verbatim}

and confirm the plan to remove any resources configured in our \texttt{.tf} files from our
OpenStack project.

More information is available at 
\begin{itemize}
  \item https://www.terraform.io/language (information about writing Terraform configurations)
  \item https://www.terraform.io/cli (information about the CLI)
\end{itemize}  


\end{document}
