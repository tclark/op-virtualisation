% Beamer slide template prepared by Tom Clark <tom.clark@op.ac.nz>
% Otago Polytechnic
% Dec 2012

\documentclass[10pt]{beamer}
\usetheme{Dunedin}
\usepackage{graphicx}
\usepackage{fancyvrb}

\newcommand\codeHighlight[1]{\textcolor[rgb]{1,0,0}{\textbf{#1}}}

\title{Kubernetes}

\author[ID720]{Virtualisation}
\institute[Otago Polytechnic]{
  Otago Polytechnic \\
  Dunedin, New Zealand \\
}
\date{}
\begin{document}

%----------- titlepage ----------------------------------------------%
\begin{frame}[plain]
  \titlepage
\end{frame}



\begin{frame}
  \frametitle{Kubernetes}
  Kubernetes 
  \begin{itemize}
    \item joins multiple machines into a cluster;
    \item lets us interact with the cluster as a single entity;
    \item allows us to run containers and set up volumes for them on the cluster;
    \item uses a container runtime (typically, but not necessarily, Docker);
    \item \textbf{adds a significant amount of extra machinery.}
  \end{itemize}
\end{frame}

\begin{frame}
  \frametitle{Kubernetes clusters}
  
  \begin{itemize}
    \item Clusters are made up of a control plane and worker nodes.
    \item Bother require a collection of software components.
    \item The components are distinct for each.
  \end{itemize}
\end{frame}

\begin{frame}
  \frametitle{Worker node components}
  
  \begin{itemize}
    \item kubelet
    \item kube-proxy
    \item a container runtime, e.g., Docker
  \end{itemize}
  
  In general we don't interact with worker nodes.
\end{frame}

\begin{frame}
  \frametitle{Control plane components}
  
  \begin{itemize}
    \item kube-api-server
    \item etcd
    \item kube-scheduler
    \item controller managers
  \end{itemize}
  
\end{frame}

\begin{frame}
  \frametitle{Add ons}
  
  \begin{itemize}
    \item DNS
    \item Web dashboard
    \item Resource monitoring
    \item Logging service
  \end{itemize}
  
  None of these is strictly required, although we generally always need DNS.
\end{frame}

\begin{frame}
  \frametitle{Admin interface}
  
  kubectl is the standard CLI tool used to administer a Kubernetes cluster. It interacts with the kube-api-server on the master. Typically we would install it on a workstation.
  
\end{frame}

\begin{frame}
  \frametitle{Installing Kubernetes}
  
  Since Kubernetes is actually a collection of software spread across several machines, there is no one standard installation method or package.
  
  We will use a special version of Kubernetes designed to run on a single machine, called \texttt{microk8s}.
  
\end{frame}

\begin{frame}
  \frametitle{Pods}
  
  \begin{itemize}
    \item The basic unit of deployment is the \emph{pod}.
    \item A pod is a collection of containers and volumes.
    \item All of the items in a pod will be deployed onto the same node.
    \item Other deployment elements are built up from pods.
  \end{itemize}
\end{frame}

\begin{frame}
  \frametitle{Pod manifests}
  
  \begin{itemize}
    \item There is more than one way to create a pod. Typically we create them from manifest files.
    \item These are JSON or YAML files (typically we use YAML).
    \item We use these files with kubectl (e.g. \texttt{kubectl apply -f mypod.yml}).
  \end{itemize}
\end{frame}

\begin{frame}[fragile]

\begin{verbatim}
apiVersion: v1
kind: Pod
metadata:
  name: kuard
spec:
  volumes:
  - name: "kuard-data"
      hostPath:
        path: "/var/lib/kuard"
  containers:
    - image: gcr.io/kuar-demo/kuard-amd64:1
      name: kuard
      volumeMounts:
        - mountPath: "/data"
          name: "kuard-data"
      ports:
        - containerPort: 8080
          name: http
          protocol: TCP
\end{verbatim}

\tiny{From: Kelsey Hightower, Brendan Burns, and Joe Beda. "Kubernetes: Up and Running."}


\end{frame}

\begin{frame}
  \frametitle{Other deployment elements}
  
  \begin{itemize}
    \item ReplicaSets
    \item DaemonSets
    \item Deployments
    \item Jobs
    \item Services
    \item Secrets
    \item ConfigMaps
  \end{itemize}
\end{frame}





\end{document}
