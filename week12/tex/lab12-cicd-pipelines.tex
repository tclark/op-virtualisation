\documentclass{article}
\usepackage{graphicx}
\usepackage{wrapfig}
%\usepackage{inconsolata}
\usepackage{enumerate}
\usepackage{hyperref}
\usepackage{verbatim}
\usepackage[parfill]{parskip}
\usepackage[margin = 2.5cm]{geometry}

\usepackage[T1]{fontenc}


\begin{document}

\title{Lab 12: CI/CD Pipelines\\ID720 Virtualisation}
\date{}
\maketitle

\section*{Introduction}
For the past few sessions we've been building up to the idea of a CI/CD pipeline. Containers themselves are useful for CI/CD, but the value of containers plus Kubernetes is that they facilitate CI/CD. This week we will set up a full pipeline

To start today's lab, create a GCP project named ``lab12''. 

\section{Google Cloud Tutorial}

When I sat down to write today's lab, I found out that somebody had already done it, so let's use it: \url{https://cloud.google.com/architecture/app-development-and-delivery-with-cloud-code-gcb-cd-and-gke?hl=en}.


\section{Clean up}
This lab uses a number of billable resources, so be sure to remove everything after you finish. 
 
\end{document}